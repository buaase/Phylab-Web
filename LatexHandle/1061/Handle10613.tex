\documentclass[11pt,a4paper,oneside]{article}
\usepackage[UTF8,adobefonts]{ctex}

\usepackage{wrapfig}
\usepackage{indentfirst}
\usepackage{amsmath}
\usepackage{float}
\usepackage{ulem}
\usepackage[top=1in,bottom=1in,left=1.25in,right=1.25in]{geometry}
\usepackage{amssymb}
\usepackage{color}
\usepackage{xcolor}

\usepackage{multirow}

\begin{document}

\subsection*{实验3.物距像距法测凹透镜焦距}

\begin{center}

\begin{tabular}{|c|c|c|c|c|}
\hline 
屏1/mm &  凹透镜1/mm & 凹透镜2/mm & 屏2/mm 均值/mm \\ 
\hline 



%% i %%

& %% i %%


\\
\hline

\end{tabular}
\vspace{10pt}

\end{center}
u = ${x}_{\text{屏1}} - {x}_{\text{均}}$\\
${u}_1$ = %% U[0] %% mm		${u}_2$ = %% U[1] %% mm		${u}_3$ = %% U[2] %% mm\\
v = $\left | {x}_{\text{屏2}} - {x}_{\text{均}} \right |$\\
${v}_1$ = %% V[0] %% mm		${v}_2$ = %% V[1] %%	mm		${v}_3$ = %% V[2] %% mm\\
$\because f = \frac{uv}{u+v}$\\
$\therefore {f}_1 = \frac{{u}_1{v}_1}{{u}_1+{v}_1}$ = %% F[0][0] %% mm	$ {f}_2$ = %% F[0][1] %% mm	$ {f}_3$ = %% F[0][2] %% mm\\
$\therefore \bar{f} = \frac{{f}_1+{f}_2+{f}_3}{3}$ = %% AVERAGE_F %% mm\\
\end{document}
