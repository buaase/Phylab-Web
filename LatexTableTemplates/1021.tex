\documentclass{article}
\usepackage{CJK}
\usepackage{multirow}
\title{测量冰的溶解热(1021)}

\begin{document}
\begin{CJK}{GBK}{song}

\maketitle

%测量冰的溶解热
\begin{tabular}{|c|c|c|c|c|}
\hline
\multicolumn{5}{|c|}{测量冰的溶解热}\\
\hline
\multicolumn{5}{|c|}{质量测量(单位:g)}\\
\hline
筒&搅拌棒&筒+水&加冰后总质量&环境温度阻值\\
\hline
\hline
\end{tabular}

\begin{tabular}{|c|c|c|c|c|c|c|c|c|c|c|}
\hline
\multicolumn{11}{|c|}{首先隔60s读数的填在此处(可不填完)}\\
\hline
序号&1&2&3&4&5&6&7&8&9&10\\
\hline
阻值&
\hline
序号&11&12&13&14&15&16&17&18&19&20\\
\hline
阻值&
\hline
\multicolumn{11}{|c|}{然后隔60s读数的填在此处(可不填完)}\\
\hline
序号&1&2&3&4&5&6&7&8&9&10\\
\hline
阻值&
\hline
序号&11&12&13&14&15&16&17&18&19&20\\
\hline
阻值&
\hline
\multicolumn{11}{|c|}{再次隔60s读数的填在此处(可不填完)}\\
\hline
序号&1&2&3&4&5&6&7&8&9&10\\
\hline
阻值&
\hline
序号&11&12&13&14&15&16&17&18&19&20\\
\hline
阻值&
\hline
\end{tabular}

%焦耳法测量热功当量
\begin{tabular}{|c|c|c|c|c|}
\hline
\multicolumn{5}{|c|}{焦耳法测热功当量}\\
\hline
内筒质量(g)& &V_{1}(v)&环境温度阻值\\
\hline
总质量(g)& &V_{2}(v)& \\
\hline
\end{tabular}

\begin{tabular}{|c|c|c|c|c|c|c|c|c|c|c|}
\hline
时间/min&0&1&2&3&4&5&6&7&8&9\\
\hline
阻值/K\Omega&
\hline
时间/min&10&11&12&13&14&15&16&17&18&19\\
\hline
阻值/K\Omega&
\hline
时间/min&20&21&22&23&24&25&26&27&28&29\\
\hline
阻值/K\Omega&
\hline
\end{tabular}

\end{CJK}{GBK}{song}
\end{document}