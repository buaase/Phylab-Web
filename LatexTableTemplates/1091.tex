\documentclass{article}
\usepackage{CJK}
\usepackage{multirow}
\title{迈克尔逊干涉(1091)}

\begin{document}
\begin{CJK}{GBK}{song}

\maketitle

%迈克尔逊干涉
\begin{tabular}{|c|c|c|c|c|c|c|c|c|c|c|}
\hline
\multicolumn{11}{|c|}{迈克尔逊干涉}\\
\hline
位置&初始&100&200&300&400&500&600&700&800&900\\
\hline
d/mm&
\end{tabular}

%牛顿环干涉
\begin{tabular}{|c|c|c|c|c|c|c|c|c|c|c|}
\hline
\multicolumn{11}{|c|}{牛顿环干涉}\\
\hline
环数&11&12&13&14&15&16&17&18&19&20\\
\hline
左/mm&
\hline
右/mm&
\hline
D/mm&0&0&0&0&0&0&0&0&0&0\\
\hline
\end{tabular}

%劈尖干涉
\begin{tabular}{|c|c|c|c|c|c|c|}
\hline
\multicolumn{7}{|c|}{}\\
\hline
&1&2&3&4&5&平均\\
\hline
L左&
\hline
L右
\hline
L&0&0&0&0&0&0\\
\hline
\end{tabular}

\begin{tabular}{|c|c|c|c|c|c|c|c|c|c|c|c|}
\hline
序号&0&0+5&0+10&0+15&0+20&0+25&0+30&0+35&0+40&0+45&0+50\\
\hline
\end{tabular}


\end{CJK}{GBK}{song}
\end{document}