\documentclass[11pt,a4paper,oneside]{article}
\usepackage[UTF8,adobefonts]{ctex}

\usepackage{wrapfig}
\usepackage{indentfirst}
\usepackage{amsmath}
\usepackage{float}
\usepackage{ulem}

\usepackage[top=1in,bottom=1in,left=1.25in,right=1.25in]{geometry}

\usepackage{color}
\usepackage{xcolor}

\usepackage{multirow}

\begin{document}

\begin{figure}[H]
 \centering
  \includegraphics[width=13cm]{Image/表头.png}
\end{figure}
\begin{center}
\textbf{{\large 实验名称:\uline{          模拟示波器的使用     }}}
\end{center}

\section*{一、实验目的}
\begin{enumerate}
\item  了解示波器的主要结构和波形显示及参数测量的基本原理,掌握示波器、信号发生器的使用方法;
\item  学习用示波器观察波形以及测量电压、周期和频率的方法;
\item  学会用连续波方法测量空气速度,加深对共振、相位等概念的理解;
\item 用示波器研究电信号谐振频率、二极管的伏安特性曲线、同轴电缆中电信号传播速度等测量方法。
\end{enumerate}

\section*{二、实验原理}
\subsection*{1.模拟示波器简介}

模拟示波器是利用电子示波管的特性,将人眼无法直接观测的交变电信号转换成图像并显示在荧光屏上以便测量和分析的电子仪器。它主要由阴极射线示波管,扫描、触发系统,放大系统,电源系统四部分组成。

\subsubsection*{(1)工作原理}
模拟示波器的基本工作原理是:被测信号经Y轴衰减后送至Y1放大器,经延迟级后到Y2放大器,信号放大后加到示波管的Y轴偏转板上。
若Y轴所加信号为图所示的正弦信号,X输入开关S切换到“外”输入,且X轴没有输入信号,则光点在荧光屏竖直方向上按正弦规律上下运动,随着Y轴方向信号的提高,由于视觉暂留,在荧光屏上显示一条竖直扫描线。同理,如在X轴所加信号为锯齿波信号,且Y轴没有输入信号,则光点在荧光屏上显示一条水平直线。

正弦信号

锯齿波信号
李萨如图形
当X轴和Y轴同时有频率相同或者成整数比的两个正弦电压输入,此时电子束同时受到两个方向偏转电压的作用,在荧光屏上的光点将显示两个正交谐振动得合成振动图形,即李萨如图形。

不难理解,沿着这种闭合轨道环绕一周后在水平和竖直方向往返的次数与两个方向的频率成正比。即李萨如图形与水平线相交的点数nx及与垂直线相交的点数ny之间的比值与两信号频率之比有如下关系:
fy: fx=nx: ny


\subsubsection*{(2)结构与使用方法}
\subsubsection*{双踪示波器面板}

\begin{description}
\item[开机]
显示屏右下方为“电源”开关,按下为开,弹出为关。调节“辉度”和“聚焦”旋钮,使光迹亮度适中且最清晰。注意:不可将光点和扫描线调得过亮。否则不仅会使眼睛疲劳,而且长时间停留会使屏幕变黑。
\item[垂直偏转因数和微调]
垂直偏转因数(VOLTS/DIV)分12挡,最高为5V/DIV,最低位1mV/DIV。使用垂直“微调旋钮”可连续改变垂直偏转因数。
\item[水平偏转因数和微调]
水平偏转因数表示锯齿波扫描的速率,共20挡,0.1$\mu$s/DIV-0.5s/DIV,利用水平“微调旋钮”同样可以连续改变水平偏转因数。
\item[工作方式的选择]
“方式”开关置于CH1,屏幕仅显示1通道的信号;置于CH2仅显示2通道的信号;“双踪”是指屏幕显示双踪,以交替或断续方式同时显示1通道和2通道的信号;“叠加”是指显示1通道和2通道信号之和。
\item[触发方式]
“自动”为扫描电路自动进行扫描,在没有信号输入或信号没有被触发同步时,屏幕上仍然可以显示扫描线;“常态”则只有触发信号才能扫描,否则屏幕上无扫描线显示。
\item[输入信号耦合]
输入信号与示波器有3种连接方式:AC(交流)为交流耦合,按钮处于弹出位置,信号经电容输入,其直流成分被阻隔;DC(直流)为直流耦合,按钮处于按入位置,信号与示波器直接耦合;另有一GND(接地)按钮,按入时输入信号与示波器断开,示波器内部输入端接地。该按钮通常用于准确测量基准或寻迹。
\end{description}
\subsection*{2.示波器的应用}
\subsubsection*{(1)电压的测量}
由于电子束在显示屏上偏转的距离与输入电压成正比,所以只要量出被测波形任意两点的垂直间距y就可知两点间的电压u,即
$$u=Ky$$
其中,K为灵敏度,也称垂直偏转系数。
如测电压为简谐波,则只要量出电压波形峰-峰的间距y,就可以知其电压的有效值U,即    
$$U=Ky/(2*2^(1/2))$$
\subsubsection*{(2)时间的测量}
用示波器可直观地测量时间。当扫描电压用锯齿波时,荧光屏上X轴坐标与时间直接相关,信号从波形上某点传至另一点所用时间t,等于该两点间距l乘以观测时的每格扫描时间t0,即t=lt0
\subsubsection*{(3)声速的测量}
$S_1$发出的声波传播到接收器后,在激发起$S_2$振动的同时又被S2的断面所反射。保持接收器端面与发射器端面平行,声波将在两平行平面之间往返反射。因为声波在换能器中的传播速度和换能器的密度都比空气要大得多,可以认为这是一个以两端刚性平面为界的空气柱的振动问题。共振条件:
                    $$l0=nλ/2$$
在$S_2$处于不同的共振位置时,各电信号极大值之间的距离均为$\displaystyle\frac{\lambda}{2}$。当接收器沿声波传播方向由近而远移动时,只要测出各极大值所对应的接收器位置,就可以测出波长$\lambda$。

\section*{三、实验仪器}
同轴电缆信号转播速度测试仪、声速测量仪、信号发生器、示波器、屏蔽电缆若干、温度计。

\section*{四、实验步骤}
\section*{实验1.模拟示波器的使用}
a.示波器预置。调节示波器的“辉度”、“聚焦”、“水平位移”、“垂直位移”等旋钮,按下触发方式的“自动”按钮,使屏上出现细而清晰的扫描线。

b.利用示波器观察其左下角的“校准信号”,校正偏转系数。

c.将正弦发生器f2和f4信号分别接入CH1通道,分别计算出电压有效值U和频率f,并绘出波形。

d.观察李萨如图形,用李萨如图形测量正弦信号频率。
\section*{实验3.声速测量}
a.连接好装置图,仔细调节频率,使示波器上的图形振幅在某一频率范围内达到最大,记下示波器上的频率。实验的最后也需要再一次记录示波器上的频率。

b.用振幅法测量波长。为了提高精度,要求测定连续10个间隔为30*λ/2的距离,接着,继续移动接收器,默数极大值到第31个时再连续测出10个极大值的位置。由上面的20个数据用逐差法计算出声波的波长及其不确定度。

c.计算声速并计算其不确定度。其中波长测量的不确定度包括3个分量:逐差法计算的A类分量,仪器误差带入的B类分量以及位置判断不准确而产生的B类分量。最后将其合成声速的不确定度。

\end{document}